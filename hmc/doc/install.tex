The software ships with a GNU autoconf environment and a configure
script, which will generate GNU Makefiles to build the programmes. It
is supported and recommended to configure and build the executables in
a separate build directory. This also allows to have several builds with
different options from the same source code directory. 

\subsection{Prerequisites}

In order to compile the programmes the {\ttfamily
  LAPACK}~\cite{lapack:web} library (fortran version) needs to be
installed. In addition it must be known which linker options are
needed to link against {\ttfamily LAPACK}, e.g. {\ttfamily
  -Lpath-to-lapack -llapack  -lblas}. Also a the latest
version (tested is version 1.2.3) of {\ttfamily
  C-LIME}~\cite{lime:web} must be available, which is used as a
packaging scheme to read and write gauge configurations and
propagators to files.

\subsection{Configuring the hmc package}
\label{sec:config}

In order to get a simple configuration of the hmc package it is enough
to just type 
\begin{verbatim}
patch-to-src-code/configure   --with-lime=<path-to-lime> \
     --with-lapack=<linker-flags> CC=<mycc> \
     F77=<myf77> CFLAGS=<c-compiler flags>
\end{verbatim}
in the build directory. If 
{\ttfamily CC, F77} and {\ttfamily CFLGAS} are not specified,
{\ttfamily configure} will guess them.

The code was successfully compiled and run at least on the following
platforms: i686 and compatible, x64 and compatible, IBM Regatta
systems, IBM Blue Gene/L, IBM Blue Gene/P, SGI Altix and SGI PC
clusters, powerpc clusters.

The configure script accepts certain options to influence the building
procedure. One can get an overview over all supported options with
{\ttfamily configure --help}. There are {\ttfamily enable|disable}
options switching on and off optional features and {\ttfamily
  with|without} switches usually related to optional packages. In the
following we describe the most important of them (check {\ttfamily
  configure --help} for the defaults and more options):

\begin{itemize}
\item {\ttfamily --enable-mpi}:\\
  This option switches on the support for MPI. On certain platforms it
  automatically chooses the correct parallel compiler or searches for
  a command {\ttfamily mpicc} in the search path.

\item {\ttfamily --enable-p4}:\\
  Enable the use of special Pentium4 instruction set and cache
  management.

\item {\ttfamily --enable-opteron}:\\
  Enable the use of special opteron instruction set and cache
  management.

%\item {\ttfamily --enable-sse}:\\
%  Enable the use of SSE instruction set. This means not much when 64
%  Bit precision is used.

\item {\ttfamily --enable-sse2}:\\
  Enable the use of SSE2 instruction set. This is a huge improvement
  on Pentium4 and equivalent systems.

\item {\ttfamily --enable-sse3}:\\
  Enable the use of SSE3 instruction set. This will give another 20\%
  of speedup when compared to only SSE2. However, only a few
  processors are capable of SSE3 so far.

\item {\ttfamily --enable-gaugecopy}:\\
  See section \ref{sec:dirac} for details on this option. It will
  increase the memory requirement of the code.

\item {\ttfamily --enable-halfspinor}:\\
  If this option is enabled the Dirac operator using half spinor
  fields is used. See sub-section \ref{sec:dirac} for details. If this
  feature is switched on, also the gauge copy feature is switched
  on automatically. 

%\item {\ttfamily --enable-shmem}:\\
%  Use shared memory API instead of MPI for the communication of spinor
%  fields. This is currently only usable on the Munich Altix machine.

\item {\ttfamily --with-mpidimension=n}:\\
  This option has only effect if the preceding one is switched
  on. The number of parallel directions can be specified. 1,2,3 and 4
  dimensional parallelisation is supported.

\item {\ttfamily --with-lapack="<linker flags>"}:\\
  the code requires lapack to be linked. All linker flags necessary
  to do so must be specified here. Note, that {\ttfamily LIBS="..."}
  works similar.

\item {\ttfamily --with-limedir=<dir>}:\\
  Tells configure where to find the lime package, which is required for
  the build of the HMC. It is used for the ILDG file format.
 
\end{itemize}

The configure script will guess at the very beginning on which
platform the build is done. In case this fails or a cross compilation
must be performed please use the option {\ttfamily --host=HOST}. For
instance in order to compile for the BG/P one needs to specify
{\ttfamily --host=ppc-ibm-bprts --build=ppc64-ibm-linux}. 

For certain architectures like the Blue Gene systems there are
{\ttfamily README.arch} files in the top source directory with
example configure calls.

\subsection{Building and Installing}

After successfully configuring the package the code can be build by
simply typing {\ttfamily make} in the build directory. This will
compile the standard executables. Typing {\ttfamily make install} will
copy these executables into the install directory. The default install
directory is {\ttfamily \$HOME/bin}, which can be influenced e.g. with
the {\ttfamily --prefix} option to {\ttfamily configure}. 


%%% Local Variables: 
%%% mode: latex
%%% TeX-master: "main"
%%% End: 
