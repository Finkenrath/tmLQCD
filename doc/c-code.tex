\subsection{C-Code}

\noindent {\ttfamily Hopping\_Matrix(int ieo, int l, int k)}:\\
Files: {\ttfamily Hopping\_Matrix.c|h}.\\
Here the Hopping Matrix is implemented.
\[
\kappa\sum_\mu \delta_{x,y+\hat\mu}(1+\gamma_\mu)
\]
It connect even (odd) sites in {\ttfamily l} with odd  (even) sites in {\ttfamily k}
if {\ttfamily ieo} is set to {\ttfamily EO} ({\ttfamily
  OE}). {\ttfamily EO} and {\ttfamily OE} are defined in {\ttfamily
  Hopping\_Matrix.h}.

\noindent {\ttfamily mul\_one\_pm\_imu\_inv(int l, double sign)}:\\
Files: {\ttfamily tm\_operators.c}.\\
This multiplies {\ttfamily l} with 
\[
l = (M_{ee}^{\pm})^{-1} l\, ,
\]
where the sign of $M_{ee}^{\pm}$ is set corresponding to {\ttfamily
  sign}. This connects even with even or odd with odd sites.

\noindent{\ttfamily mul\_one\_pm\_imu\_sub\_mul\_gamma5(int l, int k, int j,
double sign)}:\\
Files: {\ttfamily tm\_operators.c}.\\
This performs:
\[
l = \gamma_5 ((1 \pm i \mu \gamma_5)k - j)\, ,
\]
where the sign is set corresponding to {\ttfamily sign}.

\noindent{\ttfamily mul\_one\_pm\_imu\_sub\_mul(int l, int k, int j,
double sign)}:\\
Files: {\ttfamily tm\_operators.c}.\\
This performs:
\[
l = ((1 \pm i \mu \gamma_5)k - j)\, ,
\]
where the sign is set corresponding to {\ttfamily sign}.

\noindent{\ttfamily Qtm\_plus\_psi(int l, int k)}:\\
Files: {\ttfamily tm\_operators.c|h}.\\
This implements:
\[
\hat Q_{+} = \gamma_5(M_{oo}^+ - M_{oe}(M_{ee}^+ )^{-1}M_{eo})\, .
\]
The Operator is applied to {\ttfamily k} and the result is stored in
{\ttfamily l}. $\hat Q_{-}$ is implemented in {\ttfamily
  Qtm\_minus\_psi(int l, int k)}.

\noindent{\ttfamily Mtm\_plus\_psi(int l, int k)}:\\
Files: {\ttfamily tm\_operators.c|h}.\\
This implements:
\[
\hat \gamma_5Q_{+} = (M_{oo}^+ - M_{oe}(M_{ee}^+ )^{-1}M_{eo})\, .
\]
The Operator is applied to {\ttfamily k} and the result is stored in
{\ttfamily l}. $\hat \gamma_5Q_{-}$ is implemented in {\ttfamily
  Mtm\_minus\_psi(int l, int k)}.

\noindent{\ttfamily Qtm\_pm\_psi(int l, int k)}:\\
Files: {\ttfamily tm\_operators.c|h}.\\
This implements:
\[
\hat Q_{+} \hat Q_{-}\, .
\]
The Operator is applied to {\ttfamily k} and the result is stored in
{\ttfamily l}.

\noindent{\ttfamily H\_eo\_tm\_inv\_psi(int l, int k, int ieo, double
  sign)}:\\
Files:  {\ttfamily tm\_operators.c|h}.\\
This implements:
\[
l = (M_{ee|oo}^\pm)^{-1} M_{eo|oe} k\, .
\]
The sign is set corresponding to {\ttfamily sign}. Setting {\ttfamily
  ieo} to {\ttfamily EO} ({\ttfamily OE}) means using $M_{eo}$
($M_{oe}$) and $M_{ee}$ ($M_{oo}$).

\subsection{$\gamma$ matrices}

$\gamma_5$ ist defined as follows:
\[
  \gamma_5 =
  \begin{pmatrix}
    +1 & 0 & 0 & 0 \\
    0 & +1 & 0 & 0 \\
    0 & 0 & -1 & 0 \\
    0 & 0 & 0 & -1 \\    
  \end{pmatrix}\ .
\]

In the operator the following notation for
the matrices is used:
\[
\begin{split}
  \gamma_0 = \begin{pmatrix}
    0 & 0 & +1 & 0 \\
    0 & 0 & 0 & +1 \\
    +1 & 0 & 0 & 0 \\
    0 & +1 & 0 & 0 \\
  \end{pmatrix},\quad
  \gamma_1 =\begin{pmatrix}
    0 & 0 & 0 & +i \\
    0 & 0 & +i & 0 \\
    0 & -i & 0 & 0 \\
    -i & 0 & 0 & 0 \\    
  \end{pmatrix},\\
  \gamma_2 = \begin{pmatrix}
    0 & 0 & 0 & +1 \\
    0 & 0 & -1 & 0 \\
    0 & -1 & 0 & 0 \\
    +1 & 0 & 0 & 0 \\   
  \end{pmatrix},\quad
  \gamma_3 =\begin{pmatrix}
    0 & 0 & +i & 0 \\
    0 & 0 & 0 & -i \\
    -i & 0 & 0 & 0 \\
    0 & +i & 0 & 0 \\
  \end{pmatrix}\ .\\
\end{split}
\]

\subsection{boundaries in the parallel code}

The gauge field ist stored with the local gauge field first with index
{\ttfamily 0 to T*L*L*L-1}, then the right boundary with index
{\ttfamily T*L*L*L to (T+1)*L*L*L-1} and then the left boundary with index
{\ttfamily (T+1)*L*L*L to (T+2)*L*L*L-1}. This means there is no
Even-Odd order in the gauge fields. 

The spinor fields have only size {\ttfamily (VOLUME+RAND)/2}. Thus
there are either only even or only odd sites and the corresponding
boundary stored. There are two index arrays called {\ttfamily trans1}
and {\ttfamily trans2} which map from the whole volume to the even odd
counting and vice versa. {\ttfamily trans2[ix]} returns the lexical
index where {\ttfamily ix} is a even odd index and
{\ttfamily trans1[ix]} returns the even odd index where {\ttfamily ix}
is now a lexical index. 

With the index array {\ttfamily g\_ipt[t][x][y][z]} one can map the
point $(t,x,y,z)$ to the lexical index. The indices of the next
neighbours of point with lexical index {\ttfamily ix} in positiv and
negative direction $\mu$ you can get with
{\ttfamily g\_iup[ix][$\mu$]} and {\ttfamily g\_idn[ix][$\mu$]}
respectively.


%%% Local Variables: 
%%% mode: latex
%%% TeX-master: "main"
%%% End: 
