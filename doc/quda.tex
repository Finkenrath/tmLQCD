\subsection{QUDA}


The QUDA interface is complementary to tmLQCD's own CUDA kernels for computations on the GPU by Florian Burger.

\subsubsection{Design goals of the interface}
The QUDA interface has been designed with the following goals in mind, sorted by priority:
\begin{enumerate}
	\item \emph{Safety.} Naturally, highest priority is given to the correctness of the output of the interface. 
	This is trivially achieved by always checking the final residual on the CPU with the default tmLQCD routines.
	\item \emph{Ease of use.} Replacing the operator declarations
	\begin{itemize}
		\item {\ttfamily TMWILSON}
		\item {\ttfamily WILSON}
		\item {\ttfamily DBTMWILSON}
		\item {\ttfamily CLOVER}
		\item {\ttfamily DBCLOVER}
	\end{itemize}
	in the input file with their QUDA counterparts {\ttfamily TMWILSONQUDA, WILSONQUDA, DBTMWILSONQUDA, CLOVERQUDA, DBCLOVERQUDA} will let QUDA perform the inversion of that operator.
	\item \emph{Minimality.} Minimal additions in the form of {\ttfamily \#ifdef QUDA} precompiler directives to the tmLQCD code base. The main bulk of the interface lies in a separate file {\ttfamily quda\_interface.c} (with corresponding header file). A single precompiler directive block had to be added to {\ttfamily operators.c}.
	\item \emph{Performance.} The higher priority of the previous items results in small performance detriments. In particular:
	\begin{itemize}
		\item tmLQCD's $\theta$-boundary conditions disallow QUDA's 12 parameter reconstruction of the gauge fields (as of QUDA-0.7.0). Choosing to use trivial (anti-)periodic BC within {\ttfamily quda\_interface.c} overcomes this limitation but the final CPU residual check with tmLQCD routines cannot be passed then. Therefore the default is to not use 12 parameter reconstruction.
		\item The gaugefield is transferred each time to the GPU before the inversion starts (this is negligible).
	\end{itemize}
\end{enumerate}


\subsubsection{Installation}


