\section{Implementing Deflation}

We are aiming to solve 
\[
D\psi = \eta
\]
using L{\"u}schers deflation method. Note that $D$ is the
non-hermitian (twisted) Wilson Dirac operator.

Lets assume we have devided the whole lattice completely into blocks
$\Lambda(\vec b)$ on a four dimensional grid. Every block has a grid
coordinate $\vec b$ and we have a total number of $N_b$ blocks. (This
is actually what we have in the MPI environment) Lets
assume we have found $N$ approximate (global) eigenvectors $\psi_l,
l=1,...,N$ and we have restricted them to the blocks via
\[
\psi_l^{\vec b}(x) =
\begin{cases}
  \psi_l(x) & \textrm{if}\ x \in \Lambda(\vec b)\, ,\\
  0 & \textrm{otherwise}\,.
\end{cases}
\]
obtaining in total $N_b\cdot N$ fields. And we have already
orthonormalised them using Gram-Schmidt or whatever.

\subsubsection*{Construction of the little Dirac operator}

The \emph{little Dirac operator} $A$ is then computed from
\[
A_{(\vec a,k)(\vec b, l)} = (\psi_k^{\vec a},D\psi_l^{\vec b})
\]
$A_{(\vec a,k)(\vec b, l)}$ is non-zero only for $\vec a = \vec b$ or
$\vec b = \vec a \pm \vec \mu, \mu=1,...,4$, where $\vec\mu$ is a unit
vector in block space, because $D$ involves only next neighbour
interaction. 

All elements with $\vec a = \vec b$ can be computed by applying the
local Dirac operator $D^{\vec a}$ (i.e. all exterior boundaries set to
zero, because $\psi_l^{\vec a}$ has support only on block
$\Lambda(\vec a)$)
\[
\phi_l^{\vec a} = D \psi_l^{\vec a} = D^{\vec a} \psi_l^{\vec a},\quad
l=1,...,N 
\]
and computing the scalar products $(\psi_l^{\vec a},\phi_k^{\vec a})$
for all combination of $l,k$ then. For the terms with $\vec a
\neq \vec b$ we have to be more carefully. Probably it is best done by
looping over all directions $\pm\mu$ and computing
\[
(\psi_l^{\vec a},\phi_k^{\vec a+\vec\mu})_{\partial_{\mu}\Lambda(\vec a)}
\]
where $\partial_{\mu}\Lambda(\vec a)$ denotes the inner boundary in
$\mu$-direction of block $\Lambda(\vec a)$, where $\phi_k^{\vec
  a+\vec\mu}$ is non-zero. (This requires a boundary exchange in
direction $\mu$ only.) In this way we could come away with one
application of $D$ per global field $\psi_l$.

The action of the little Dirac operator on a \emph{little
  quark field} $w$ on block $\Lambda(\vec a)$ reads:
\[
\begin{split}
  v_k^{\vec a} &= A_{(\vec a,k)(\vec b, l)} w_l^{\vec b}\\
  &= \sum_{l=1}^N  A_{(\vec a,k)(\vec a, l)} w_l^{\vec a} + 
  \sum_{\mu} \sum_{l=1}^N [A_{(\vec a,k)(\vec a+\vec\mu, l)} w_l^{\vec
    a+\vec\mu} + A_{(\vec a,k)(\vec a-\vec\mu, l)} w_l^{\vec a - \vec\mu}]\\
\end{split}
\]
or in matrix notation
\[
v^{\vec a} = A^{\vec a,\vec a} w^{\vec a} + \sum_\mu [A^{\vec a,\vec
  a+\vec\mu} w^{\vec a+\vec\mu} +  A^{\vec a,\vec a-\vec\mu} w^{\vec a-\vec\mu}]
\]
This involves again only next neighbour (block) interaction. What is
the best data type for the little fields $v,w$? We need exchange
routines for them. Every vector $w^{\vec a}$ has $N$ complex entries
and every $A^{\vec a,\vec b}$ is $N\times N$ complex matrix. 

\subsubsection{global mode deflation}

We want to use the global fields $\psi_l$ to deflate the little Dirac
operator $A$. This requires to find vectors $\chi_l$, which fulfill
\begin{equation}
  \label{eq:llD}
  B_{kl} = \langle\psi_k|D \psi_l\rangle = \langle\chi_k|A \chi_l\rangle
\end{equation}
for $k,l = 1,...,N_s$. We recall that the fields $\phi_i$, $i=
1,...,N_b\cdot N_s$ were obtained by restricting the fields $\psi_l$ to
the blocks $\vec a$ followed by an orthonormalisation process. So the
fields $\chi_l^{\vec a}$ on block $\vec a$ are computed from
\begin{equation}
  \label{eq:chis}
  (\chi^{\vec a}_l)_i = \langle\phi_i|\psi_l^{\vec a}\rangle,
\end{equation}
where the notation $(v^{\vec a})_i$ means the $i$-th component of
vector $v$ on block $\vec a$. The little little Dirac operator $B$ is
then given by
\[
\begin{split}
  B_{kl} &= \langle\chi_k|A \chi_l\rangle =
  \langle(\chi_k)_i|\langle\phi_i|D \phi_j\rangle (\chi_l)_j\rangle\\
  &= \langle\psi_k|\phi_i\rangle\langle\phi_i|D
  \phi_j\rangle\langle\phi_j |\psi_l\rangle = \langle\psi_k|D \psi_l\rangle \\
\end{split}
\]
where summing over equal indices is understood. The little Dirac
operators is then deflated using the oblique projectors
\[
\begin{split}
  p_L v &= v - \sum_{k,l}^{N_s}
  A\chi_k(B^{-1})_{kl}\langle\chi_l|v\rangle\\
  p_R v &= v - \sum_{k,l}^{N_s}
  \chi_k(B^{-1})_{kl}\langle\chi_l|A v\rangle\\
\end{split}
\]
and the same algebra as before.

\subsubsection*{What needs to be done}

\begin{itemize}
\item implement a suitable preconditioner
\item implement an eigenvector generator, possibly using the
  preconditioner 
\item implement the construction of $A$
\item implement the action of $A$ on a little vector $w$
\item implement an iterative solver for $w=A^{-1}v$, do we need
  even/odd preconditioning here? Probably we should use BiCGstab.
\item implement the oblique projectors $P_L$ and $P_R$. (equations 3.3
  and 3.4 in L{\"u}schers paper)
\item implement the construction of the final solution from the
  solution of the deflated system and the little Dirac operator (eq
  3.10) 
\end{itemize}
%%% Local Variables: 
%%% mode: latex
%%% TeX-master: "main"
%%% End: 
