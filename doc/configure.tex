\section{Configuring the hmc package}

In order to get a simple configuration of the hmc package it is enough
to just call {\ttfamily ./configure} in the top level directory. The
c-compiler choice is influenced e.g. by specifying {\ttfamily
  ./configure CC=<mycc>}. 

The configure script accepts certain options to influence the building
procedure. One can get an overview over all supported options with
{\ttfamily configure --help}. There are {\ttfamily enable|disable}
options switching on and off optional features and {\ttfamily
  with|without} switches usually related to optional packages. In the
following we describe the most important of them (check {\ttfamily
  configure --help} for the defaults):

\begin{itemize}
\item {\ttfamily --enable-mpi}:\\
  This option switches on the support for MPI. On certain platforms it
  automatically chooses the correct parallel compiler or searches for
  a command {\ttfamily mpicc} in the searchpath.

\item {\ttfamily --enable-p4}:\\
  Enable the use of special petium4 instruction set and cache
  management.

\item {\ttfamily --enable-sse}:\\
  Enable the use of SSE instruction set. This means not much when 64
  Bit precision is used.

\item {\ttfamily --enable-sse2}:\\
  Enable the use of SSE2 instruction set. This is a huge improvement
  on pentium4 and equivalent systems.

\item {\ttfamily --enable-sse3}:\\
  Enable the use of SSE3 instruction set. This will give another 20\%
  of speedup when compared to only SSE2. However, only a few
  processors are capable of SSE3 so far.

\item {\ttfamily --enable-gaugecopy}:\\
  See section \ref{subsec:storage} for details on this option.

\item {\ttfamily --enable-eogeom}:\\
  See section \ref{subsec:storage} for details on this option.


\item {\ttfamily --with-mpidimension=x}:\\
  This option has only effect if the preceeding one is switched
  on. The number of parallel direction can be specified. Currently
  {\ttfamily x=1} and {\ttfamily x=2} are supported and implemented,
  corresponding to parallelisation in time- or in time- and x-direction,
  respectively.

\item {\ttfamily --with-lapack}:\\
  Build the HMc with lapack. If this is disabled the Chronological
  solver guess cannot be used. Istead zero spinor fields will be used
  as initial guesses for the solvers during the molecular dynamics.

\item {\ttfamily --with-limedir=<dir>}:\\
 Tells configure where to find the lime package, which is required for
 the build of the HMC. It is used for the ILDG file format.

\end{itemize}

The configure script will guess at the very beginning on which
platform the build is done. In case this failes or a cross compilation
must be performed please use the option {\ttfamily --host=HOST}. For
instance in order to compile for the BGL you have to specify
{\ttfamily --host=powerpc64-bgl-linux-gnu}.

Special values for {\ttfamily CC, CFLAGS, LDFLAGS} can be given to
configure as demonstrated above for {\ttfamily CC}.


%%% Local Variables: 
%%% mode: latex
%%% TeX-master: "main"
%%% End: 
