% $Id$
\label{sec:eo}

\subsection{HMC Update}

In this section we describe how even/odd
\cite{DeGrand:1990dk,Jansen:1997yt} preconditioning can be used in the
HMC algorithm in 
presence of a twisted mass term. Even/odd preconditioning is
implemented in the tmLQCD package in the HMC algorithm as well as in
the inversion of the Dirac operator, and can be used optionally.

We start with the lattice fermion action in the hopping parameter
representation in the $\chi$-basis written as
\begin{equation}
  \label{eq:eo0}
    \begin{split}
    S[\chi,\bar\chi,U] = \sum_x & \Biggl\{ \bar\chi(x)[1+2i \kappa\mu\gamma_5\tau^3]\chi(x)  \Bigr. \\
    & -\kappa\bar\chi(x)\sum_{\mu = 1}^4\Bigl[ U(x,\mu)(r+\gamma_\mu)\chi(x+a\hat\mu)\bigr. \\
    & +\Bigl. \bigl. U^\dagger(x-a\hat\mu,\mu)(r-\gamma_\mu)\chi(x-a\hat\mu)\Bigr]
    \Biggr\} \\
    \equiv &\sum_{x,y}\bar\chi(x) M_{xy}\chi(y)\ .
  \end{split}
\end{equation}
For convenience we define
$\tilde\mu=2\kappa\mu$. Using the matrix $M$ one can define the
hermitian (two flavor) operator:
\begin{equation}
  \label{eq:eo1}
  Q\equiv \gamma_5 M = \begin{pmatrix}
    \Qp & \\\
    & \Qm \\
  \end{pmatrix}
\end{equation}
where the sub-matrices $\Qpm$ can be factorised as follows (Schur
decomposition): 
\begin{equation}
  \label{eq:eo2}
  \begin{split}
    Q^\pm &= \gamma_5\begin{pmatrix}
      1\pm i\tilde\mu\gamma_5 & M_{eo} \\
      M_{oe}    & 1\pm i\tilde\mu\gamma_5 \\
    \end{pmatrix} =
    \gamma_5\begin{pmatrix}
      M_{ee}^\pm & M_{eo} \\
      M_{oe}    & M_{oo}^\pm \\
    \end{pmatrix} \\
    & =
    \begin{pmatrix}
      \gamma_5M_{ee}^\pm & 0 \\
      \gamma_5M_{oe}  & 1 \\
    \end{pmatrix}
    \begin{pmatrix}
      1       & (M_{ee}^\pm)^{-1}M_{eo}\\
      0       & \gamma_5(M_{oo}^\pm-M_{oe}(M_{ee}^\pm)^{-1}M_{eo})\\
    \end{pmatrix}\, .
\end{split}
\end{equation}
Note that $(M_{ee}^\pm)^{-1}$ can be
computed to be 
\begin{equation}
  \label{eq:eo3}
  (1\pm i\tilde\mu\gamma_5)^{-1} = \frac{1\mp i\tilde\mu\gamma_5}{1+\tilde\mu^2}.
\end{equation}
Using $\det(Q)=\det(\Qp)\det(\Qm)$ the following relation can be derived
\begin{equation}
  \label{eq:eo4}
  \begin{split}
    \det(\Qpm) &\propto \det(\hQpm) \\
    \hQpm &= \gamma_5(M_{oo}^\pm - M_{oe}(M_{ee}^\pm )^{-1}M_{eo})\, ,
  \end{split}
\end{equation}
where $\hQpm$ is only defined on the odd sites of the lattice. In the
HMC algorithm the determinant is stochastically estimated using pseudo
fermion field $\phi_o$: Now we write the determinant with pseudo
fermion fields:
\begin{equation}
  \begin{split}
    \det(\hQp \hQm) &= \int \mathcal{D}\phi_o\,\mathcal{D}\phi^\dagger_o\
    \exp(-S_\mathrm{PF})\\
    S_\mathrm{PF} &\equiv\ \phi_o^\dagger\ \left(\hQp\hQm\right)^{-1}\phi_o\, ,
  \end{split}
\end{equation}
where the fields $\phi_o$ are defined only on the odd sites of the
lattice. In order to compute the force corresponding to the effective
action $S_\mathrm{PF}$ we need the variation of $S_\mathrm{PF}$ with respect to the gauge fields
(using $\delta (A^{-1})=-A^{-1}\delta A A^{-1}$):
\begin{equation}
  \label{eq:eo5}
  \begin{split}
    \delta S_\mathrm{PF} &= -[\phi_o^\dagger (\hQp \hQm)^{-1}\delta \hQp(\hQp)^{-1}\phi_o +
    \phi_o^\dagger(\hQm)^{-1}\delta \hQm (\hQp \hQm)^{-1} \phi_o ] \\
     &= -[X_o^\dagger \delta \hQp Y_o + Y_o^\dagger \delta\hQm X_o]
  \end{split}
\end{equation}
with $X_o$ and $Y_o$ defined on the odd sides as 
\begin{equation}
  \label{eq:eo6}
  X_o = (\hQp \hQm)^{-1} \phi_o,\quad Y_o = (\hQp)^{-1}\phi_o=\hat
  \Qm X_o\ ,
\end{equation}
where $(\hQpm)^\dagger = \hat Q^\mp$ has been used. The variation of
$\hQpm$ reads
\begin{equation}
  \label{eq:eo7}
  \delta \hQpm = \gamma_5\left(-\delta M_{oe}(M_{ee}^\pm )^{-1}M_{eo} -
    M_{oe}(M_{ee}^\pm )^{-1}\delta M_{eo}\right),
\end{equation}
and one finds
\begin{equation}
  \label{eq:eo8}
  \begin{split}
    \delta S_\mathrm{PF} &= -(X^\dagger\delta \Qp Y + Y^\dagger\delta \Qm X) \\
    &= -(X^\dagger\delta \Qp Y +(X^\dagger\delta \Qp Y)^\dagger)
  \end{split}
\end{equation}
where $X$ and $Y$ are now defined over the full lattice as
\begin{equation}
  \label{eq:eo9}
  X = 
  \begin{pmatrix}
    -(M_{ee}^-)^{-1}M_{eo}X_o \\ X_o\\
  \end{pmatrix},\quad
  Y = 
  \begin{pmatrix}
    -(M_{ee}^+)^{-1}M_{eo}Y_o \\ Y_o\\
  \end{pmatrix}.
\end{equation}
In addition $\delta\Qp = \delta\Qm, M_{eo}^\dagger = \gamma_5 M_{oe}\gamma_5$ and
$M_{oe}^\dagger = \gamma_5 M_{eo}\gamma_5$ has been used. Since the bosonic part
is quadratic in the $\phi_o$ fields, the $\phi_o$ are generated at the
beginning of each molecular dynamics trajectory with
\begin{equation}
  \label{eq:eo10}
  \phi_o = \hQp R,
\end{equation}
where $R$ is a random spinor field taken from a Gaussian distribution
with norm one.

\subsubsection{Symmetric even/odd Preconditioning}

One may write instead of eq. (\ref{eq:eo2}) the following symmetrical
factorisation of $\Qpm$:
\begin{equation}
  \label{eq:sym1}
  \Qpm =
  \gamma_5\begin{pmatrix}
    M_{ee}^\pm & 0 \\
    M_{oe}  &  M_{oo}^\pm \\
  \end{pmatrix}
  \begin{pmatrix}
    1       & (M_{ee}^\pm)^{-1}M_{eo}\\
    0       & (1-(M_{oo}^\pm)^{-1} M_{oe} (M_{ee}^\pm)^{-1} M_{eo})\\
  \end{pmatrix}\, .
\end{equation}
Where we can now re-define
\begin{equation}
  \label{eq:sym2}
  \hat Q_\pm = \gamma_5(1-(M_{oo}^\pm)^{-1} M_{oe} (M_{ee}^\pm)^{-1}
  M_{eo}) 
\end{equation}
With this re-definition the procedure is analogous to what we
discussed previously. Only the vectors $X$ and $Y$ need to be modified
to  
\begin{equation}
  \begin{split}
    \label{eq:eo9}
    X &= 
    \begin{pmatrix}
      -(M_{ee}^-)^{-1}M_{eo}(M_{oo}^-)^{-1}X_o \\ X_o\\
    \end{pmatrix},\\
    Y &= 
    \begin{pmatrix}
      -(M_{ee}^+)^{-1}M_{eo}(M_{oo}^+)^{-1}Y_o \\ Y_o\\
    \end{pmatrix}.\\
  \end{split}
\end{equation}
Note that the variation of the action is still given by
\begin{equation}
  \label{eq:sym3}
  \delta S_\mathrm{PF} = -\re(X^\dagger \delta Q_+ Y)\ .
\end{equation}

\subsubsection{Mass non-degenerate flavour doublet}

Even/odd preconditioning can also be implemented for the mass
non-degenerate flavour doublet Dirac operator $D_h$
eq.~(\ref{eq:Dh}). Denoting 
\[
Q^h = \gamma_5 D_h
\]
the even/odd decomposition is as follows
\begin{equation}
  \label{eq:Dheo}
  \begin{split}
    Q^h &=
    \begin{pmatrix}
      (\gamma_5+i\bar\mu\tau^3 -\bar\epsilon\tau^1) & Q^h_{eo}\\
      Q^h_{oe} & (\gamma_5+i\bar\mu\tau^3 -\bar\epsilon\tau^1)\\
    \end{pmatrix} \\
    &=
    \begin{pmatrix}
      Q^h_{ee} & 0 \\
      Q^h_{oe} & 0 \\
    \end{pmatrix}
    \cdot
    \begin{pmatrix}
      1 & (Q^h_{ee})^{-1}Q_{eo} \\
      0 & Q^h_{oo} \\
    \end{pmatrix} \\
  \end{split}
\end{equation}
where $Q^h_{oo}$ is given in flavour space by
\begin{equation*}
  Q^h_{oo} = \gamma_5
  \begin{pmatrix}
    1 + i\bar\mu\gamma_5 -
    \frac{M_{oe}(1-i\bar\mu\gamma_5)M_{eo}}{1+\bar\mu^2-\bar\epsilon^2} & 
    -\bar\epsilon\left(1+\frac{M_{oe}M_{eo}}{1+\bar\mu^2-\bar\epsilon^2}\right) \\
    -\bar\epsilon\left(1+\frac{M_{oe}M_{eo}}{1+\bar\mu^2-\bar\epsilon^2}\right) & 
    1 - i\bar\mu\gamma_5 -
    \frac{M_{oe}(1-i\bar\mu\gamma_5)M_{eo}}{1+\bar\mu^2-\bar\epsilon^2}\\
  \end{pmatrix}
\end{equation*}
with the previous definitions of $M_{eo}$ etc. The inplementation for
the HMC is very similar to the mass degenerate case.

\subsubsection{Clover term and Twisted mass term}

We start again with the lattice fermion action in the hopping
parameter representation in the $\chi$-basis now including the clover
term written as
\begin{equation}
  \label{eq:eosw0}
    \begin{split}
    S[\chi,\bar\chi,U] = \sum_x & \Biggl\{ \bar\chi(x)[1+2\kappa
    c_{SW}T + 2i \kappa\mu\gamma_5\tau^3]\chi(x)  \Bigr. \\
    & -\kappa\bar\chi(x)\sum_{\mu = 1}^4\Bigl[ U(x,\mu)(r+\gamma_\mu)\chi(x+a\hat\mu)\bigr. \\
    & +\Bigl. \bigl. U^\dagger(x-a\hat\mu,\mu)(r-\gamma_\mu)\chi(x-a\hat\mu)\Bigr]
    \Biggr\} \\
    \equiv &\sum_{x,y}\bar\chi(x) M_{xy}\chi(y)\, ,
  \end{split}
\end{equation}
with the clover term $T$. For convenience we define
$\tilde\mu\equiv2\kappa\mu$ and $\tilde c_{SW} = 2\kappa
c_{SW}$. Using the matrix $M$ one can define the 
(two flavor) operator:
\begin{equation}
  \label{eq:eosw1}
  Q\equiv \gamma_5 M = \begin{pmatrix}
    \Qp & \\\
    & \Qm \\
  \end{pmatrix}
\end{equation}
where the sub-matrices $\Qpm$ can be factorised as follows (Schur
decomposition): 
\begin{equation}
  \label{eq:eosw2}
  \begin{split}
    Q^\pm &= \gamma_5\begin{pmatrix}
      1 + T_{ee} \pm i\tilde\mu\gamma_5 & M_{eo} \\
      M_{oe}    & 1 + T_{oo} \pm i\tilde\mu\gamma_5 \\
    \end{pmatrix} =
    \gamma_5\begin{pmatrix}
      M_{ee}^\pm & M_{eo} \\
      M_{oe}    & M_{oo}^\pm \\
    \end{pmatrix} \\
    & =
    \begin{pmatrix}
      \gamma_5M_{ee}^\pm & 0 \\
      \gamma_5M_{oe}  & 1 \\
    \end{pmatrix}
    \begin{pmatrix}
      1       & (M_{ee}^\pm)^{-1}M_{eo}\\
      0       & \gamma_5(M_{oo}^\pm-M_{oe}(M_{ee}^\pm)^{-1}M_{eo})\\
    \end{pmatrix}\, .
\end{split}
\end{equation}
Note that $(M_{ee}^\pm)^{-1}$ cannot be computed as easily as in the
case of Twisted mass fermions without clover term.
Using $\det(Q)=\det(\Qp)\det(\Qm)$ the following relation can be derived
\begin{equation}
  \label{eq:eosw4}
  \begin{split}
    \det(\Qpm) &\propto \det(1+T_{ee} \pm i\tilde\mu\gamma_5)\det(\hQpm) \\
    \hQpm &= \gamma_5(M_{oo}^\pm - M_{oe}(M_{ee}^\pm )^{-1}M_{eo})\, ,
  \end{split}
\end{equation}
where $\hQpm$ is only defined on the odd sites of the lattice. In the
HMC algorithm the second determinant is stochastically estimated using
pseudo fermion fields $\phi_o$: now we write the determinant with
pseudo fermion fields:
\begin{equation}
  \begin{split}
    \det(\hQp \hQm) &= \int \mathcal{D}\phi_o\,\mathcal{D}\phi^\dagger_o\
    \exp(-S_\mathrm{PF})\\
    S_\mathrm{PF} &\equiv\ \phi_o^\dagger\ \left(\hQp\hQm\right)^{-1}\phi_o\, ,
  \end{split}
\end{equation}
where the fields $\phi_o$ are defined only on the odd sites of the
lattice. A second term needs to be taken into account, which reads
\begin{equation}
  \label{eq:swdet}
  S_{\det} =  - \tr[\log(1+T_{ee} + i\tilde\mu\gamma_5) -
  \log(1+T_{ee} - i\tilde\mu\gamma_5)]\, .
\end{equation}
In order to compute the force corresponding to the effective
action $S_\mathrm{PF}$ we need the variation of $S_\mathrm{PF}$ with
respect to the gauge fields 
(using $\delta (A^{-1})=-A^{-1}\delta A A^{-1}$):
\begin{equation}
  \label{eq:eosw5}
  \begin{split}
    \delta S_\mathrm{PF} &= -[\phi_o^\dagger (\hQp \hQm)^{-1}\delta \hQp(\hQp)^{-1}\phi_o +
    \phi_o^\dagger(\hQm)^{-1}\delta \hQm (\hQp \hQm)^{-1} \phi_o ] \\
     &= -[X_o^\dagger \delta \hQp Y_o + Y_o^\dagger \delta\hQm X_o]
  \end{split}
\end{equation}
with $X_o$ and $Y_o$ defined on the odd sides as 
\begin{equation}
  \label{eq:eosw6}
  X_o = (\hQp \hQm)^{-1} \phi_o,\quad Y_o = (\hQp)^{-1}\phi_o=\hat
  \Qm X_o\ ,
\end{equation}
where $(\hQpm)^\dagger = \hat Q^\mp$ has been used. The variation of
$\hQpm$ reads
\begin{equation}
  \label{eq:eosw7}
  \begin{split}
    \delta \hQpm = \gamma_5 & \left( \delta T_{oo}-\delta M_{oe}(M_{ee}^\pm )^{-1}M_{eo} -
      M_{oe}(M_{ee}^\pm )^{-1}\delta M_{eo}\right. \\
  &\left. M_{oe}(M_{ee}^\pm )^{-1} \delta T_{ee} (M_{ee}^\pm )^{-1} M_{eo}\right),
  \end{split}
\end{equation}
and one finds
\begin{equation}
  \label{eq:eosw8}
  \begin{split}
    \delta S_\mathrm{PF} &= -(X^\dagger\delta \Qp Y + Y^\dagger\delta \Qm X) \\
    &= -(X^\dagger\delta \Qp Y +(X^\dagger\delta \Qp Y)^\dagger)
  \end{split}
\end{equation}
where $X$ and $Y$ are now defined over the full lattice as
\begin{equation}
  \label{eq:eosw9}
  X = 
  \begin{pmatrix}
    -(M_{ee}^-)^{-1}M_{eo}X_o \\ X_o\\
  \end{pmatrix},\quad
  Y = 
  \begin{pmatrix}
    -(M_{ee}^+)^{-1}M_{eo}Y_o \\ Y_o\\
  \end{pmatrix}.
\end{equation}
In addition $\delta\Qp = \delta\Qm, M_{eo}^\dagger = \gamma_5 M_{oe}\gamma_5$ and
$M_{oe}^\dagger = \gamma_5 M_{eo}\gamma_5$ has been used. Since the bosonic part
is quadratic in the $\phi_o$ fields, the $\phi_o$ are generated at the
beginning of each molecular dynamics trajectory with
\begin{equation}
  \label{eq:eosw10}
  \phi_o = \hQp R,
\end{equation}
where $R$ is a random spinor field taken from a Gaussian distribution
with norm one.

The additional bit in the action $S_{\det}$ needs to be treated
seperately. The variation of this part is
\begin{equation}
  \label{eq:eosw11}
  \delta S_{\det} = -\tr \left[(1+i\tilde\mu\gamma_5 + T_{ee})^{-1}  +
    (1-i\tilde\mu\gamma_5 + T_{ee})^{-1}\right] \delta T_{ee}\ . 
\end{equation}
The main difference in between pure Twisted mass fermions and Twisted
mass fermions plus clover term is that the matrices $M_{ee}$ and
$M_{oo}$ need to be inverted numerically. A stable numerical method
for this task needs to be devised.

\subsection{Inversion}

In addition to even/odd preconditioning in the HMC algorithm as
described above, it can also be used to speed up the inversion of the
fermion matrix. 

Due to the factorization (\ref{eq:eo2}) the full fermion matrix can be
inverted by inverting the two matrices appearing in the factorization
\[
\begin{pmatrix}
  M_{ee}^\pm & M_{eo} \\
  M_{oe}    & M_{oo}^\pm \\
\end{pmatrix}^{-1}
=
\begin{pmatrix}
  1       & (M_{ee}^\pm)^{-1}M_{eo}\\
  0       & (M_{oo}^\pm-M_{oe}(M_{ee}^\pm)^{-1}M_{eo})\\
\end{pmatrix}^{-1}
\begin{pmatrix}
  M_{ee}^\pm & 0 \\
  M_{oe}   & 1 \\
\end{pmatrix}^{-1}\, .
\]
The two factors can be simplified as follows:
\[
\begin{pmatrix}
  M_{ee}^\pm & 0 \\
  M_{oe}   & 1 \\
\end{pmatrix}^{-1}
=
\begin{pmatrix}
      (M_{ee}^\pm)^{-1} & 0 \\
      -M_{oe} (M_{ee}^{\pm})^{-1}  & 1 \\
    \end{pmatrix}
\]
and 
\[
\begin{split}
  &\begin{pmatrix}
    1       & (M_{ee}^\pm)^{-1}M_{eo}\\
    0       & (M_{oo}^\pm-M_{oe}(M_{ee}^\pm)^{-1}M_{eo})\\
  \end{pmatrix}^{-1}
  \\=&
  \begin{pmatrix}
    1       & -(M_{ee}^\pm)^{-1}M_{eo}(M_{oo}^\pm-M_{oe}(M_{ee}^\pm)^{-1}M_{eo})^{-1}  \\
    0       & (M_{oo}^\pm-M_{oe}(M_{ee}^\pm)^{-1}M_{eo})^{-1}\\
  \end{pmatrix}\, .
\end{split}
\]
The complete inversion is now performed in two separate steps: First
we compute for a given source field $\phi=(\phi_e,\phi_o)$ an intermediate 
result $\varphi=(\varphi_e,\varphi_o)$ by:
\[
\begin{pmatrix}
  \varphi_e \\ \varphi_o\\
\end{pmatrix}
=
\begin{pmatrix}
  M_{ee}^\pm & 0 \\
  M_{oe}   & 1 \\
\end{pmatrix}^{-1}
\begin{pmatrix}
  \phi_e \\ \phi_o \\
\end{pmatrix}
=
\begin{pmatrix}
  (M_{ee}^\pm)^{-1} \phi_e \\ 
  -M_{oe}( M_{ee}^\pm)^{-1} \phi_e + \phi_o \\
\end{pmatrix}\, .
\]
This step requires only the application of $M_{oe}$ and
$(M_{ee}^\pm)^{-1}$, the latter of which is given by Eq~(\ref{eq:eo3}).
The final solution $\psi=(\psi_e,\psi_o)$ can then be computed with
\[
\begin{pmatrix}
  \psi_e \\ \psi_o \\
\end{pmatrix}
=
\begin{pmatrix}
  1       & (M_{ee}^\pm)^{-1}M_{eo}\\
  0       & (M_{oo}^\pm-M_{oe}(M_{ee}^\pm)^{-1}M_{eo})\\
\end{pmatrix}^{-1}
\begin{pmatrix}
  \varphi_e \\ \varphi_o \\
\end{pmatrix}
=
\begin{pmatrix}
  \varphi_e - (M_{ee}^\pm)^{-1}M_{eo}\psi_o \\ \psi_o \\
\end{pmatrix}\, ,
\]
where we defined
\[
\psi_o = (M_{oo}^\pm-M_{oe}(M_{ee}^\pm)^{-1}M_{eo})^{-1} \varphi_o\, .
\]
Therefore the only inversion that has to be performed numerically is
the one to generate $\psi_o$ from $\varphi_o$ and this inversion
involves only an operator that is better conditioned than the original
fermion operator.

Even/odd preconditioning can also be used for the mass non-degenerate
Dirac operator $D_h$ eq.~(\ref{eq:Dh}). The corresponding equations
follow immediately from the previous discussion and the definition
from eq.~(\ref{eq:Dheo}).

\subsubsection{Inverting $M$ on $\phi_o$}

In case inverting the full matrix $M$ is much faster than inverting
the even/odd preconditioned matrix -- as might be the case with
deflation, one may use for symmetric even/odd preconditioining
\begin{equation}
  (\hat M^\pm)^{-1}\phi_o\ =\ P_{l\to o}\ (M_\pm)^{-1}\ P_{o\to l}\
  M^\pm_{oo}\ \phi_o
\end{equation}
Where $P_{l\to o}$ projects the odd sides of a full spinor and
$P_{o\to l}$ reverses this by filling up with zeros. $M_\pm$ is here just
$\gamma_5 Q_\pm$. For asymmetric even/odd preconditioning the formula
reads
\begin{equation}
  (\hat M^\pm)^{-1}\phi_o\ =\ P_{l\to o}\ (M_\pm)^{-1}\ P_{o\to l}\
  \phi_o\, .
\end{equation}
It is based on the observation that
\[
M^{-1} = 
\begin{pmatrix}
  A_{ee} & A_{eo} \\
  A_{oe} & A_{oo} \\
\end{pmatrix}
\]
with (skipping the $\pm$ index for brevity)
\[
\begin{split}
  A_{ee}\quad &=\quad (1- M_{ee}^{-1} M_{eo} M_{oo}^{-1} M_{oe})^{-1}\ M_{ee}^{-1} \\
  A_{eo}\quad &=\quad -M_{ee}^{-1}\ M_{eo}\ A_{oo} \\
  A_{oe}\quad &=\quad -M_{oo}^{-1}\ M_{oe}\ A_{ee} \\
  A_{oo}\quad &=\quad (1- M_{oo}^{-1} M_{oe} M_{ee}^{-1} M_{eo})^{-1}\ M_{oo}^{-1} \\
\end{split}
\]
\endinput

%%% Local Variables: 
%%% mode: latex
%%% TeX-master: "main"
%%% End: 
