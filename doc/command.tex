% $Id$
\newtheorem{theorem}{Theorem}[section]
\newtheorem{theo}{Theorem}[section]
\newtheorem{exam}{Theorem}[section]
\newtheorem{example}[exam]{Example}
\newtheorem{corollary}[theorem]{Lemma}
\newtheorem{lemma}[theorem]{Lemma}
\newtheorem{definition}[theo]{Definition}
\newtheorem{remark}[exam]{Remark}
\newtheorem{problem}[theorem]{Problem}
\newtheorem{question}[theorem]{Question}
\newtheorem{satz}[theorem]{Theorem}
\newtheorem{Aussage}[theorem]{Aussage}

\newenvironment{proof}{\noindent{\bf Proof:~}}%
{\null\hfill $\Box$\par\medskip}
\newenvironment{fixme}{\noindent{\bf FIXME:\\~}}%
{\\{\bf ENDFIXME}\par\medskip}
\newenvironment{Bemerkung}{\noindent{\bf Remark:\\~}}%
{}

%\renewcommand{\theequation}{\thesection-\arabic{equation}}
%\renewcommand{\thefigure}{\thesection.\arabic{figure}}
\renewcommand{\topfraction}{1}
\renewcommand{\bottomfraction}{1}
\renewcommand{\textfraction}{0}
%\renewcommand{\chaptermark}[1]%
%{\markboth{\chaptername\ \thechapter\ #1}{}}
\renewcommand{\sectionmark}[1]%
{\markright{\thesection\ #1}}

%Spur, Real- und Imaginaerteil
\newcommand{\tr}{\operatorname{Tr}}
\newcommand{\re}{\operatorname{Re}}
\newcommand{\im}{\operatorname{Im}}
\newcommand{\C}{\mathbb{C}}
\newcommand{\R}{\mathbb{R}}
\newcommand{\ID}{\mathbb{I}}
\newcommand{\rf}{\mathcal{R}_5^{\mathbf{sp}}}
%\newcommand{\myinput}[1]{\def\old{\filename}%
%  \def\filename{\tiny [#1.tex,\today]}%
%  \cfoot{\filename}%
%  \input{#1}%
%  \cfoot{\old}%
%}
\newcommand{\myinput}[1]{%
  \input{#1}%
}

%229und 230 oder 235
%\newcommand{\totheright}{\ding{230}{\sffamily\scshape Rechts}: }
\newcommand{\totheright}{(b): }
%\newcommand{\toright}{\ding{230}{\sffamily\scshape Right}}
\newcommand{\toright}{b}
%\newcommand{\totheleft}{\reflectbox{\ding{229}}{\sffamily\scshape Links}: }
\newcommand{\totheleft}{(a): }
%\newcommand{\toleft}{\reflectbox{\ding{229}}{\sffamily\scshape Links}}
\newcommand{\toleft}{a}
\newcommand{\Mathlogo}{{\scshape Mathematica}\Pisymbol{psy}{226} }

%Farben von root nach dessen nummern
\definecolor{eins}{rgb}{0,0,0}%black
\definecolor{zwei}{rgb}{1,0,0}%red
\definecolor{drei}{rgb}{0,1,0}%green
\definecolor{vier}{rgb}{0,0,1}%blue
\definecolor{fuenf}{rgb}{1,1,0}%yellow
\definecolor{pink}{rgb}{1,0,1}
\definecolor{sechs}{rgb}{1,0,1}%pink
\definecolor{sieben}{rgb}{0,1,1}%cyan
\definecolor{acht}{rgb}{0.35,0.83,0.33}%darkgreen
\definecolor{neun}{rgb}{0.35,0.33,0.85}%darkblue

\def\sometext{%
Text Text Text Text Text Text Text Text Text Text Text Text Text Text Text Text Text Text Text Text %
Text Text Text Text Text Text Text Text Text Text Text Text Text Text Text Text Text Text Text Text %
Text Text Text Text Text Text Text Text Text Text Text Text Text Text Text Text Text Text Text Text %
Text Text Text Text Text Text Text Text Text Text Text Text Text Text Text Text Text Text Text Text %
}

\endinput

%%% Local Variables: 
%%% mode: latex
%%% TeX-master: "main"
%%% End: 
