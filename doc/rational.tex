\subsection{Rational HMC}

For the heavy doublet one may alternatively use a rational
approximation 
\[
\mathcal{R}(\hat Q_h^2)\ = \ \prod_{i = 1}^N \frac{\hat Q_h^2 +
  a_{2i-1}}{\hat Q_h^2 + a_{2i}}\approx\quad\frac{1}{\sqrt{\hat Q_h^2}}
\]
where we used the shorthand notation
\[
\hat Q_h^2\ =\ \gamma_5 \hat D_h \tau^1\gamma_5\hat D_h \tau^1 
\]
and $\hat Q_h=\gamma_5\hat D_h\tau^1$ is the even/odd preconditioned version
of $Q_h$ defined in Eq.~(\ref{eq:Dh}). Obviously, we have $\hat
Q_h^\dagger = \hat Q_h$. 
We are using the Zolotarev solution for the
optimal rational approximation to $1/\sqrt{y}$. The coefficients $a_i$
fulfill the property
\[
a_1 > a_2 > ... > a_{2N} > 0\, .
\]
We use the partial fraction expansion to re-express
\[
\mathcal{R}(\hat Q_h^2)\ = \ 1 + \sum_{i=1}^{N} \frac{q_i}{\hat Q_h^2 +
  \mu_i^2}\, .
\]
The coefficients $r_i$ are given as
\[
q_i = (a_{2i-1} - a_{2i}) \prod_{m=1, m\neq i}^N \frac{a_{2m-1}
  - a_{2i}}{a_{2m} - a_{2i}}\,,\quad i = 1,...,N\,.
\]
If we defined -- following L{\"u}scher -- $\mu_i = \sqrt{a_{2i}}$ and $\nu_i
= \sqrt{a_{2i-1}}$, we may rewrite $q_i$ as
\[
q_i = (\nu_i^2 - \mu_i^2)\prod_{m=1, m\neq i}^N \frac{\nu_m^2 -
  \mu_i^2}{\mu_m^2 - \mu_i^2}\,,\quad i = 1,...,N\, .
\]
For the heatbath step we need to generate pseudo-fermion fields from
Gaussian random fields $R$
\[
R^\dagger R = \phi^\dagger \mathcal{R} \phi
\]
and, therefore, we need operators $C^\dagger, C$ with
\[
\mathcal{R}^{-1} = C^\dagger\ \cdot\ C\,,\qquad \phi = C\cdot R\,.
\]
$C$ is given by (inspired by twisted mass)
\[
C\ =\ \prod_{i=1}^N \frac{\hat Q_h + i\mu_i}{\hat Q_h + i\nu_i}
\]
which can again be written as a partial fraction
\[
C\ =\ 1 + i\sum_{i=1}^N \frac{r_i}{\hat Q_h + i\nu_i}\,,
\]
with
\[
r_i = (\mu_i - \nu_i)\prod_{m=1, m\neq i}^N \frac{\mu_m -
  \nu_i}{\nu_m - \nu_i}\,,\quad i = 1,...,N\, .
\]
The rational approximation $\mathcal{R}$ can be applied to a vector
using a multi-mass solver and the partial fraction representation. The
same works for $C$: after solving $N$ equations simultaneously for
$(\hat Q_h^2 + \nu_i^2)^{-1},\quad i = 1,...,N$, we have to multiply
every term with $(\hat Q_h - i\nu_i)$. The hermitian conjugate of $C$
is given by
\[
C^\dagger\ =\ 1 - i\sum_{i=1}^N \frac{r_i}{\hat Q_h - i\nu_i}\,,
\]
using $\hat Q_h^\dagger = \hat Q_h$.

For the acceptance step one just needs an application of $\mathcal{R}$.

\subsubsection{Force Computation}

For the derivative and the force computation we have to consider terms
of the form
\[      
\phi^\dagger \frac{q_i}{\hat Q_h^2 + \mu_i^2}\phi\,,
\]
and its variation with respect to the gauge fields:
\[      
\begin{split}   
\delta_U\  \phi^\dagger \frac{q_i}{\hat Q_h^2 + \mu_i^2}\phi &=
q_i\phi^\dagger\frac{1}{\hat Q_h + i\mu_i}\frac{1}{\hat Q_h -
i\mu_i}(-\delta_U \hat Q_h)\frac{1}{\hat Q_h -i\mu_i}\phi\ +\
\textrm{h.c.}\\
&= -2 \re\left( q_i\phi^\dagger\frac{1}{\hat Q_h^2 + \mu_i^2}
(\delta_U \hat Q_h)\frac{1}{\hat Q_h -i\mu_i}\phi\ \right)
\end{split}
\]

\subsubsection{Splitting of the Rational}

For preconditioning the fermion determinant it is useful to split the
rational into several products
\[
\mathcal{R}(\hat Q_h^2)\ = r_{0}^{l}(\hat Q_h^2)\cdot r_{l}^{k}(\hat
Q_h^2)\cdot ...
\]
with terms
\[
r_{c_0}^{c_1} = \ \prod_{i = c_0}^{c_1} \frac{\hat Q_h^2 +
  a_{2i-1}}{\hat Q_h^2 + a_{2i}}\,.
\]
Every term $r_{c_0}^{c_1}$ can then again be written as a partial
fraction with the same coefficients as given above. In
Ref.~\cite{Clark:2006fx} it was shown that the different partial 
fractions contribute quite differently in their magnitude of the
corresponding force to the MD evolution: the smallest shifts and,
therefore, most expensive ones contribute the least to the
force. Hence, those can be integrated on a larger timescale than the
larger shifts, which contribute significantly more to the total MD
force.

\subsubsection{Correction Monomial}

The rational approximation has a finite precision. In the HMC one can
account for this effect by estimating
\[
1 - |\hat Q_h| R\,,
\]
which can be done in different ways:
\begin{itemize}
\item we include an additional monomial for
  \[
  \det (|\hat Q_h| R)
  \]
  in the Hamiltonian. If the rational apprximation is precise enough,
  it is sufficient to only include this in the heatbath and acceptance
  step and ignore the contribution to the derivative. For generating
  the pseudo-fermion field for this monomial, one needs to find
  \[
  B\cdot B^\dagger = |\hat Q_h| \mathcal{R},
  \]
  which, following Ref.~\cite{Luscher:2010ae}, can be expanded in
  terms of
  \[
  Z = \hat Q_h^2\mathcal{R}^2 -1\,.
  \]
  The series
  \[
  B = (1+Z)^{1/4} =  1 + \frac{1}{4} Z - \frac{3}{32} Z^2 + \frac{7}{128} Z^3 + ...
  \]
  is rapidly converging and can usually be truncated after the $Z^2$
  or latest $Z^3$ term, see
  Refs.~\cite{Luscher:2010ae,Luscher:2012av}. We then obtain the
  pseudo-fermion field $\phi$ by
  \[
  \phi = B\cdot R\,,
  \]
  where $R$ is again a random Gaussian field. For the acceptance step
  one needs to compute
  \[
  \phi^\dagger (|\hat Q_h|\mathcal{R})^{-1}\phi\,,
  \]
  which, again expanding in $Z$ is obtained by
  \[
  \phi^\dagger (1+Z)^{-1/2} \phi = \phi^\dagger (1 - \frac{1}{2}Z +
  \frac{3}{8}Z^2 - \frac{5}{16}Z^3 + ...) \phi\, .
  \]
  Also here the series can be truncated after the first few terms.
\item the second possibility is to include this correction as a
  reweighting factor.
\item the third is to use a more precise rational approximation for
  the heatbath and acceptance steps.
\end{itemize}

\subsubsection{CGMMS Solver}

\begin{algorithm}
  \caption{CGMMS algorithm}
  \label{alg:cgm}
  \begin{algorithmic}[1]
    \vspace{.2cm}
    \STATE $n=0, x_0^k = 0, r_0 = p_0 = p_0^k = b, k_\mathrm{max},
    \delta, \epsilon$
    \STATE  $\biggl.\biggr.\alpha_{-1} = \zeta_{-1}^k = \zeta_0^k = 1, \beta_0^k = \beta_0 = 0$
    \REPEAT
    \STATE $\alpha_n = (r_n, r_n) / (p_n, A p_n)$
    \FOR{$k = 1$ to $k_\mathrm{max}$}
    \STATE $\biggl.\biggr.\zeta_{n+1}^k = (\zeta^k_n  \alpha_{n-1}) / 
      (\alpha_n \beta_n(1 - \zeta_n^k / \zeta^k_{n-1}) + \alpha_{n-1}
      (1-\sigma_k\alpha_n))$
    \STATE $\alpha^k_n = (\alpha_n \zeta_{n+1}^k)/ \zeta_n^k$
    \STATE $\biggl.\biggr.x_{n+1}^k = x_n^k + \alpha_n^k p_n^k$
    \IF{$\|\alpha^{k_\mathrm{max}} p^{k_\mathrm{max}}\| < \delta$}
    \STATE $k_\mathrm{max} = k_\mathrm{max} -1$
    \ENDIF
    \ENDFOR
    \STATE $x_{n+1} = x_n + \alpha_n p_n$
    \STATE $\biggl.\biggr.r_{n+1} = r_n - \alpha_n Ap_n$
    \STATE $\beta_{n+1} = (r_{n+1}, r_{n+1}) / (r_n, r_n)$
    \STATE $\beta_{n+1}^k = \frac{\beta_{n+1} \zeta_{n+1}^k \alpha_n^k}{\zeta_{n}^k\alpha_n}$
    \STATE $\biggl.\biggr.p_{n+1}^k = \zeta_{n+1}^k r_{n+1} + \beta_{n+1}^k p_n^k$
    \STATE $n=n+1$
    \UNTIL{$\|r_n\|<\epsilon$}
  \end{algorithmic}
\end{algorithm}


For evaluating the rational approximation $\mathcal{R}$ applied to a
spinor field $\psi$ a multi-mass or multi-shift solver (see
algorithm~\ref{alg:cgm}) can be used, see Ref.~\cite{Chiarappa:2006hz}
and references therein. However, a little care is needed
as the shift vary over several orders of magnitudes.

The original Krylov space is build for the shift smallest in
modulus. This will converge slowest and, therefore, the other shifts
will have the same or better precision guaranteed. But, if the
range in the shifts is too large, one needs to remove the highest
shifts in the course of the CG solve before the smallest shift is
converged. This will prevent the appearance of double precision
underflow and hence the appearance of exact zeros in
$\zeta^{k_\mathrm{max}}$, which would lead to 
NaNs in the solution vectors.

In order to avoid to compute the residue for all the shift frequently
during the CGMMS solve, one can rather monitor the norm of the
correction vector $p^\sigma$ of the currently biggest shift $\sigma$
still in the process. The CG works such that the correction decreases
with decreasing residue. Therefore, one can remove the shift $\sigma$
when
\[
\|\alpha^\sigma p^\sigma\| < \delta\,,
\]
where one could for instance chose $\delta =
c\cdot\epsilon$. $\epsilon$ is the desired precision of the CGMMS
solve and $0<c\leq1$ some suitably chosen constant. Removing converged
solutions has the side effect of speeding up the CGMMS solve in terms
of computing time.

