\subsection{Rational HMC}

For the heavy doublet one may alternatively use a rational
approximation 
\[
\mathcal{R}(\hat Q_h^2)\ = \ \prod_{i = 1}^N \frac{\hat Q_h^2 +
  a_{2i-1}}{\hat Q_h^2 + a_{2i}}\approx\quad\frac{1}{\sqrt{\hat Q_h^2}}
\]
where we used the shorthand notation
\[
\hat Q_h^2\ =\ \gamma_5 \hat D_h \tau^1\gamma_5\hat D_h \tau^1 
\]
and $\hat Q_h=\gamma_5\hat D_h\tau^1$ is the even/odd preconditioned version
of $Q_h$ defined in Eq.~(\ref{eq:Dh}). Obviously, we have $\hat
Q_h^\dagger = \hat Q_h$. 
We are using the Zolotarev solution for the
optimal rational approximation to $1/\sqrt{y}$. The coefficients $a_i$
fulfill the property
\[
a_1 > a_2 > ... > a_{2N} > 0\, .
\]
We use the partial fraction expansion to re-express
\[
\mathcal{R}(\hat Q_h^2)\ = \ 1 + \sum_{i=1}^{N} \frac{q_i}{\hat Q_h^2 +
  \mu_i^2}\, .
\]
The coefficients $r_i$ are given as
\[
q_i = (a_{2i-1}^2 - a_{2i}^2) \prod_{m=1, m\neq i}^N \frac{a_{2m-1}^2
  - a_{2i}^2}{a_{2m}^2 - a_{2i}^2}\,,\quad i = 1,...,N\,.
\]
If we defined -- following L{\"u}scher -- $\mu_i = \sqrt{a_{2i}}$ and $\nu_i
= \sqrt{a_{2i-1}}$, we may rewrite $q_i$ as
\[
q_i = (\nu_i^2 - \mu_i^2)\prod_{m=1, m\neq i}^N \frac{\nu_m^2 -
  \mu_i^2}{\mu_m^2 - \mu_i^2}\,,\quad i = 1,...,N\, .
\]
For the heatbath step we need to generate pseudo-fermion fields from
Gaussian random fields $R$
\[
R^\dagger R = \phi^\dagger \mathcal{R} \phi
\]
and, therefore, we need operators $C^\dagger, C$ with
\[
\mathcal{R}^{-1} = C^\dagger\ \cdot\ C\,,\qquad \phi = C\cdot R\,.
\]
$C$ is given by (inspired by twisted mass)
\[
C\ =\ \prod_{i=1}^N \frac{\hat Q_h + i\mu_i}{\hat Q_h + i\nu_i}
\]
which can again be written as a partial fraction
\[
C\ =\ 1 + i\sum_{i=1}^N \frac{r_i}{\hat Q_h + i\nu_i}\,,
\]
with
\[
r_i = (\mu_i - \nu_i)\prod_{m=1, m\neq i}^N \frac{\mu_m -
  \nu_i}{\nu_m - \nu_i}\,,\quad i = 1,...,N\, .
\]
The rational approximation $\mathcal{R}$ can be applied to a vector
using a multi-mass solver and the partial fraction representation. The
same works for $C$: after solving $N$ equations simultaneously for
$(\hat Q_h^2 + \nu_i^2)^{-1},\quad i = 1,...,N$, we have to multiply
every term with $(\hat Q_h - i\nu_i)$. The hermitian conjugate of $C$
is given by
\[
C^\dagger\ =\ 1 - i\sum_{i=1}^N \frac{r_i}{\hat Q_h - i\nu_i}\,,
\]
using $\hat Q_h^\dagger = \hat Q_h$.

\subsubsection{Force Computation}

For the derivative and the force computation we have to consider terms
of the form
\[      
\phi^\dagger \frac{q_i}{\hat Q_h^2 + \mu_i^2}\phi\,,
\]
and its variation with respect to the gauge fields:
\[      
\begin{split}   
\delta_U\  \phi^\dagger \frac{q_i}{\hat Q_h^2 + \mu_i^2}\phi &=
q_i\phi^\dagger\frac{1}{\hat Q_h + i\mu_i}\frac{1}{\hat Q_h -
i\mu_i}(-\delta_U \hat Q_h)\frac{1}{\hat Q_h -i\mu_i}\phi\ +\
\textrm{h.c.}\\
&= -2 \re\left( q_i\phi^\dagger\frac{1}{\hat Q_h^2 + \mu_i^2}
(\delta_U \hat Q_h)\frac{1}{\hat Q_h -i\mu_i}\phi\ \right)
\end{split}
\]

\subsubsection{Splitting of the Rational}

For preconditioning the fermion determinant it is useful to split the
rational into several products
\[
\mathcal{R}(\hat Q_h^2)\ = r_{0}^{l}(\hat Q_h^2)\cdot r_{l}^{k}(\hat
Q_h^2)\cdot ...
\]
with terms
\[
r_{c_0}^{c_1} = \ \prod_{i = c_0}^{c_1} \frac{\hat Q_h^2 +
  a_{2i-1}}{\hat Q_h^2 + a_{2i}}\,.
\]
Every term $r_{c_0}^{c_1}$ can then again be written as a partial
fraction with the same coefficients as given above. In
Ref.~\cite{Clark:2006fx} it was shown that the different partial 
fractions contribute quite differently in their magnitude of the
corresponding force to the MD evolution: the smallest shifts and,
therefore, most expensive ones contribute the least to the
force. Hence, those can be integrated on a larger timescale than the
larger shifts, which contribute significantly more to the total MD
force.
