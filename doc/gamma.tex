\label{sec:gammas}

In the following we specify our conventions for $\gamma$- and
Pauli-matrices. 

\subsection{$\gamma$-matrices}

We use the following convention for the Dirac $\gamma$-matrices:
\[
\begin{split}
  \gamma_0 = \begin{pmatrix}
    0 & 0 & -1 & 0 \\
    0 & 0 & 0 & -1 \\
    -1 & 0 & 0 & 0 \\
    0 & -1 & 0 & 0 \\
  \end{pmatrix},\quad
  \gamma_1 = \begin{pmatrix}
    0 & 0 & 0 & -i \\
    0 & 0 & -i & 0 \\
    0 & +i & 0 & 0 \\
    +i & 0 & 0 & 0 \\    
  \end{pmatrix},\\
  \gamma_2 = \begin{pmatrix}
    0 & 0 & 0 & -1 \\
    0 & 0 & +1 & 0 \\
    0 & +1 & 0 & 0 \\
    -1 & 0 & 0 & 0 \\   
  \end{pmatrix},\quad
  \gamma_3 = \begin{pmatrix}
    0 & 0 & -i & 0 \\
    0 & 0 & 0 & +i \\
    +i & 0 & 0 & 0 \\
    0 & -i & 0 & 0 \\
  \end{pmatrix}\ .\\
\end{split}
\]
In this representation $\gamma_5$ is diagonal and reads
\[
  \gamma_5 =
  \begin{pmatrix}
    +1 & 0 & 0 & 0 \\
    0 & +1 & 0 & 0 \\
    0 & 0 & -1 & 0 \\
    0 & 0 & 0 & -1 \\    
  \end{pmatrix}\ .
\]

\subsection{Pauli-matrices}

For the Pauli-matrices acting in flavour space we use the following
convention: 
\[
\begin{split}
  1_f = 
  \begin{pmatrix}
    1 & 0 \\
    0 & 1 \\
  \end{pmatrix},\quad
  \tau^1 =
  \begin{pmatrix}
    0 & 1 \\
    1 & 0 \\
  \end{pmatrix},\quad
  \tau^2 = 
  \begin{pmatrix}
    0 & -i \\
    i & 0 \\
  \end{pmatrix},\quad
  \tau^3 = 
  \begin{pmatrix}
    1 & 0 \\
    0 & -1 \\
  \end{pmatrix}
\end{split}
\]

\endinput

%%% Local Variables: 
%%% mode: latex
%%% TeX-master: "main"
%%% End: 
