\section{Hasenbusch trick for dynamical tmQCD}

Let $\hQpm$ and $\hWpm$ be two matrices as defined in (\ref{eq:eo4})
with two parameters $\mu_1$ and $\mu_2$, respectively. The idea is to
choose $\mu_2$ bigger than $\mu_1$. With this we can
write
\begin{equation}
  \label{eq:mt0}
  \det[\hQp\hQm] = \det[\hWp\hWm]\cdot\det[\hWp^{-1}\hQp\hQm\hWm^{-1}].
\end{equation}
The first term on the right hand side of (\ref{eq:mt0}) can be handled
as described in the previous section. The second term needs some
further investigation: we again write the determinant as an integral
over pseudo fermion fields:
\begin{equation}
  \label{eq:mt1}
  \begin{split}
    \det[\hWp^{-1}\hQp\hQm\hWm^{-1}]  &\propto \int
    D[\phi_o]D[\phi_o^\dagger]\exp(-\phi_o^\dagger (\hWp^{-1}\hQp\hQm\hWm^{-1})^{-1}
    \phi_o) \\
    &= \int D[\phi_o]D[\phi_o^\dagger]\exp(-\phi_o^\dagger \hWm\hQm^{-1}\hQp^{-1}\hWp\phi_o) \\
    &= \int D[\phi_o]D[\phi_o^\dagger]\exp(-S_{F_2})
  \end{split}
\end{equation}
The variation of $S_{F_2}$, needed for the HMC, then reads as follows:
\begin{equation}
  \label{eq:mt2}
  \begin{split}
    \delta S_{F_2} &= \phi_o^\dagger[\delta\hWm(\hQp\hQm)^{-1}\hWp
    +\hWm(\hQp\hQm)^{-1}\delta\hWp]\phi_o\\
    &-\phi_o^\dagger[\hWm\hQm^{-1}\delta\hQm(\hQp\hQm)^{-1}\hWp +
    \hWm(\hQp\hQm)^{-1}\delta\hQp\hQp^{-1}\hWp]\phi_o
  \end{split}
\end{equation}
If we define now
\begin{equation}
  \label{eq:mt3}
  X_W = (\hQp\hQm)^{-1}\hWp\phi\, ,\quad Y_W = \hQp^{-1}\hWp\phi = \hQm
  X_W\, ,
\end{equation}
we can rewrite (\ref{eq:mt2}):
\begin{equation}
  \label{eq:mt4}
  \begin{split}
    \delta S_{F_2} &= \phi_o^\dagger\delta\hWm X_W + X_W^\dagger\delta\hWp\phi_o\\
    &-Y_W^\dagger\delta\hQm X_W - X_W^\dagger\delta\hQp Y_W\, .
  \end{split}
\end{equation}
Recalling the variation of $\hQpm$ (and of $\hWpm$):
\begin{equation}
  \label{eq:mt5}
  \begin{split}
    \delta\hQpm &= \gamma_5\left(-\delta M_{oe}(1\pm i\mu_1\gamma_5 )^{-1}M_{eo} -
    M_{oe}(1\pm i\mu_1\gamma_5 )^{-1}\delta M_{eo}\right)\, , \\
  \delta \hWpm &= \gamma_5\left(-\delta M_{oe}(1\pm i\mu_2\gamma_5 )^{-1}M_{eo} -
    M_{oe}(1\pm i\mu_2\gamma_5)^{-1}\delta M_{eo}\right)\, ,
  \end{split}
\end{equation}
we find:
\begin{equation}
  \label{eq:mt6}
  \begin{split}
    \delta S_{F_2} &= Y_2^\dagger \delta Q X_2 + X_2^\dagger\delta QY_2 -X_1^\dagger\delta Q Y_1 -
    Y_1^\dagger\delta Q X_1\\
    &= 2\re\left[Y_2^\dagger \delta Q X_2 - Y_1^\dagger\delta Q X_1 \right]\, ,
  \end{split}
\end{equation}
where the fields $X_{1,2}$, $Y_{1,2}$ and the matrix $\delta Q$ are now
defined over the full lattice as follows:
\begin{equation}
  \label{eq:mt7}
  \begin{split}
    Y_1 &= 
    \begin{pmatrix}
      -(1+i\mu_{1}\gamma_5)^{-1}M_{eo}Y_W \\ Y_W\\
    \end{pmatrix}\, ,\quad
    Y_2 = 
    \begin{pmatrix}
       -(1+i\mu_{2}\gamma_5)^{-1}M_{eo}\phi_o \\ \phi_o\\
    \end{pmatrix},\\
    X_{1,2} &= 
    \begin{pmatrix}
      -(1-i\mu_{1,2}\gamma_5)^{-1}M_{eo}X_W \\ X_W\\
    \end{pmatrix},\quad
    \delta Q = \gamma_5
    \begin{pmatrix}
      0 & \delta M_{eo}\\
      \delta M_{oe} & 0\\
    \end{pmatrix}\, .
  \end{split}
\end{equation}
The bosonic part is again quadratic in the fields $\phi_o$ and can be
therefore generated at the beginning of each molecular dynamics
trajectory with:
\begin{equation}
  \label{eq:mt8}
  \phi_o = \hWp^{-1}\hQp R
\end{equation}
where $R$ is again a random spinor field taken from a Gaussian
distribution with norm one.
%%% Local Variables: 
%%% mode: latex
%%% TeX-master: "main"
%%% End: 
