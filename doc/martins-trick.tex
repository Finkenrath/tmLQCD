\section{Hasenbusch trick for dynamical tmQCD}

The trick presented in \cite{Hasenbusch:2002ai} is based on the
observation that writing
\[
\det[Q^2] = \det[W^2]\cdot\det[W^{-2}Q^2]
\]
is advantagous for the HMC, if the condition number of $W^{2}$ and of
$W^{-2}Q^{2}$ is significantly reduced compared to the condition number
of only $Q^2$. Thus if we define
\[
\begin{split}
  \Qpm = \gamma_5 D_W \pm i\mu_1\, ,\\
  \Wpm = \gamma_5 D_W \pm i\mu_2\, ,\\
\end{split}
\]
with $\mu_2 =\mu_1+\Delta\mu$ it follows immidiatly that the condition number of
$\Wp\Wm$ is lower than the one of $\Qp\Qm$ if for $\lambda_{\textrm{min}}$
and $\lambda_{\textrm{max}}$ the lowest and the largest eigenvalue of
$\Qp\Qm$, respectively $|\lambda_{\textrm{min}}|\ll\mu_2^2\ll|\lambda_{\textrm{max}}|$
holds: It is $|\lambda_{\textrm{max}}|/\mu_2^2$. The condition number of
$W^{-2}Q^{2}$ contrariwise is $\mu_2^2/|\lambda_{\textrm{min}}|^2$. We can take
$\mu_1$ which is a lower bound for $|\lambda_{\textrm{min}}|$ to write down
the condition numbers:
\[
k_{W^2} = \frac{|\lambda_{\textrm{max}}|}{\mu_2^2}\, ,\quad
k_{W^{-2}Q^{2}} \leq \frac{\mu_2^2}{\mu_1^2}\, .
\]
This again leads to an optimal choice for
$\mu_2^2=\sqrt{|\lambda_{\textrm{max}}|\cdot\mu_1}$. It is also illuminating to take
a look at the force coming from $W^{-2}Q^{2}$. Noticing that
\[
\Wpm = \Qpm \pm i\Delta\mu\, ,
\]
we see immidiatly that
\begin{equation}
  \label{eq:mt01}
  \Wpm^{-1}\Qpm = \mathds{1} \mp i\Delta\mu\Wpm^{-1}\, .
\end{equation}
Now we again write the determinant with pseudo fermion
fields:
\[
\begin{split}
  \det[\Wp^{-1}\Qp\Qm\Wm^{-1}] &= \int D[\phi]D[\phi^\dagger]
  \exp(-\phi^\dagger\Wm\Qm^{-1}\Qp^{-1}\Wp\phi) \\
  &=  \int D[\phi]D[\phi^\dagger] \exp(-S_F)\, .
\end{split}
\]
Using equation (\ref{eq:mt01}) we get the following expression for
$S_F$:
\begin{equation}
  \label{eq:mt02}
  S_F = \phi^\dagger(\mathds{1} +i\Delta\mu\Qp^{-1}
  -i\Delta\mu\Qm^{-1}+\Delta\mu^2(\Qp\Qm)^{-1}) \phi\, .
\end{equation}
Without explicitly computing the variation of $S_F$ with respect to
the gauge fields we can see that it can only contain term proportional
to $\Delta\mu$ and $\Delta\mu^2$, since $S_F$ is a constant up to terms of this
order. If we take $\Delta\mu$ to be small we expect a smaller force comming
from $W^{-2}Q^{2}$ than it would come from $Q^2$ and therefore a
smoother evolution of the Hamiltonian.

This we will now apply on top of even odd preconditioning. For this we
start directly with the even odd preconditioned matrices $\hQpm$, but
probably it is also possible to start from the results obtained in
this subsection. Nevertheless the above discussion was usefull to
understand a possible gain with this trick.

\subsection{Including EO preconditioning}

Let $\hQpm$ and $\hWpm$ be two matrices as defined in (\ref{eq:eo4})
with two parameters $\mu_1$ and $\mu_2$, respectively. The idea is to
choose $\mu_2$ bigger than $\mu_1$. With this we can
write
\begin{equation}
  \label{eq:mt0}
  \det[\hQp\hQm] = \det[\hWp\hWm]\cdot\det[\hWp^{-1}\hQp\hQm\hWm^{-1}].
\end{equation}
The first term on the right hand side of (\ref{eq:mt0}) can be handled
as described in the previous section. The second term needs some
further investigation: we again write the determinant as an integral
over pseudo fermion fields:
\begin{equation}
  \label{eq:mt1}
  \begin{split}
    \det[\hWp^{-1}\hQp\hQm\hWm^{-1}]  &\propto \int
    D[\phi_o]D[\phi_o^\dagger]\exp(-\phi_o^\dagger (\hWp^{-1}\hQp\hQm\hWm^{-1})^{-1}
    \phi_o) \\
    &= \int D[\phi_o]D[\phi_o^\dagger]\exp(-\phi_o^\dagger \hWm\hQm^{-1}\hQp^{-1}\hWp\phi_o) \\
    &= \int D[\phi_o]D[\phi_o^\dagger]\exp(-S_{F_2})
  \end{split}
\end{equation}
The variation of $S_{F_2}$, needed for the HMC, then reads as follows:
\begin{equation}
  \label{eq:mt2}
  \begin{split}
    \delta S_{F_2} &= \phi_o^\dagger[\delta\hWm(\hQp\hQm)^{-1}\hWp
    +\hWm(\hQp\hQm)^{-1}\delta\hWp]\phi_o\\
    &-\phi_o^\dagger[\hWm\hQm^{-1}\delta\hQm(\hQp\hQm)^{-1}\hWp +
    \hWm(\hQp\hQm)^{-1}\delta\hQp\hQp^{-1}\hWp]\phi_o
  \end{split}
\end{equation}
If we define now
\begin{equation}
  \label{eq:mt3}
  X_W = (\hQp\hQm)^{-1}\hWp\phi\, ,\quad Y_W = \hQp^{-1}\hWp\phi = \hQm
  X_W\, ,
\end{equation}
we can rewrite (\ref{eq:mt2}):
\begin{equation}
  \label{eq:mt4}
  \begin{split}
    \delta S_{F_2} &= \phi_o^\dagger\delta\hWm X_W + X_W^\dagger\delta\hWp\phi_o\\
    &-Y_W^\dagger\delta\hQm X_W - X_W^\dagger\delta\hQp Y_W\, .
  \end{split}
\end{equation}
Recalling the variation of $\hQpm$ (and of $\hWpm$):
\begin{equation}
  \label{eq:mt5}
  \begin{split}
    \delta\hQpm &= \gamma_5\left(-\delta M_{oe}(1\pm i\mu_1\gamma_5 )^{-1}M_{eo} -
      M_{oe}(1\pm i\mu_1\gamma_5 )^{-1}\delta M_{eo}\right)\, , \\
    \delta\hWpm &= \gamma_5\left(-\delta M_{oe}(1\pm i\mu_2\gamma_5 )^{-1}M_{eo} -
      M_{oe}(1\pm i\mu_2\gamma_5)^{-1}\delta M_{eo}\right)\, ,
  \end{split}
\end{equation}
we find:
\begin{equation}
  \label{eq:mt6}
  \begin{split}
    \delta S_{F_2} &= Y_2^\dagger \delta Q X_2 + X_2^\dagger\delta QY_2 -X_1^\dagger\delta Q Y_1 -
    Y_1^\dagger\delta Q X_1\\
    &= 2\re\left[Y_2^\dagger \delta Q X_2 - Y_1^\dagger\delta Q X_1 \right]\, ,
  \end{split}
\end{equation}
where the fields $X_{1,2}$, $Y_{1,2}$ and the matrix $\delta Q$ are now
defined over the full lattice as follows:
\begin{equation}
  \label{eq:mt7}
  \begin{split}
    Y_1 &= 
    \begin{pmatrix}
      -(1+i\mu_{1}\gamma_5)^{-1}M_{eo}Y_W \\ Y_W\\
    \end{pmatrix}\, ,\quad
    Y_2 = 
    \begin{pmatrix}
      -(1+i\mu_{2}\gamma_5)^{-1}M_{eo}\phi_o \\ \phi_o\\
    \end{pmatrix},\\
    X_{1,2} &= 
    \begin{pmatrix}
      -(1-i\mu_{1,2}\gamma_5)^{-1}M_{eo}X_W \\ X_W\\
    \end{pmatrix},\quad
    \delta Q = \gamma_5
    \begin{pmatrix}
      0 & \delta M_{eo}\\
      \delta M_{oe} & 0\\
    \end{pmatrix}\, .
  \end{split}
\end{equation}
The bosonic part is again quadratic in the fields $\phi_o$ and can be
therefore generated at the beginning of each molecular dynamics
trajectory with:
\begin{equation}
  \label{eq:mt8}
  \phi_o = \hWp^{-1}\hQp R
\end{equation}
where $R$ is again a random spinor field taken from a Gaussian
distribution with norm one.

This method can be applied recursively. In fact we have implemented
three pseudo fermion fields.

\subsection{Results}

\begin{table}[htbp]
  \centering
  \begin{tabular}{|c|c|c|c|c|c|}
    \hline
    scheme & $\mu_2$ & $\mu_3$ & $\delta\tau$ & stat & $P_{\textrm{acc}}$ \\
    \hline\hline
    S & $0.0$ & $0.0$ & $0.05$ & $200$ & $0.89$ \\
    \hline
    S & $0.0$ & $0.0$ & $0.1$ & $200$ & $0.32$ \\
    S & $0.034$ & $0.0$ & $0.1$ & $200$ & $0.87$ \\
    S & $0.1$ & $0.0$ & $0.1$ & $200$ & $0.91$ \\
    S & $0.2$ & $0.0$ & $0.1$ & $1000$ & $0.96$ \\
    S & $0.3$ & $0.0$ & $0.1$ & $200$ & $0.69$ \\
    S & $0.5$ & $0.0$ & $0.1$ & $200$ & $0.27$ \\
    \hline
    S & $0.2$ & $0.0$ & $0.2$ & $1000$ & $0.73$ \\
    S & $0.2$ & $0.4$ & $0.2$ & $1000$ & $0.80$\\
    S & $0.2$ & $0.6$ & $0.2$ & $1000$ & $0.80$ \\
    S & $0.2$ & $0.8$ & $0.2$ & $1000$ & $0.77$ \\
    S & $0.1$ & $0.3$ & $0.2$ & $1000$ & $0.84$ \\
    S & $0.1$ & $0.4$ & $0.2$ & $1000$ & $0.86$ \\
    S & $0.034$ & $0.18$ & $0.2$ & $1000$ & $0.77$ \\
    \hline
  \end{tabular}
  \caption{Acceptance rate for a $8^4$ Lattice at $\beta=5.2$, $\kappa=0.170$ and $\tilde\mu = 0.0034$.}
  \label{tab:resmt}
\end{table}

%%% Local Variables: 
%%% mode: latex
%%% TeX-master: "main"
%%% End: 
