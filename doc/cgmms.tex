\subsection{CGMMS}

The multi-shift CG implementation in tmLQCD is referred to as \emph{CGMMS} since it was originally developped to solve a multi-system of equations of the form
\begin{equation}
  ( A + \mathbb{I}\mu_k^2 ) = b \, ,
\end{equation}
where $A$ can be $\Qp\Qm$ or $\Mp\Mm$ and the squared shifts $\mu_i^2$ can be naturally interpreted as different twisted quark masses (in the case of $\Mp\Mm$, appropriate factors of $\gamma^5$ must be inserted as required).
\begin{equation}
  \begin{split}
  ( \Mw + i \mu \gamma^5 )( \Mw^\dagger - i \mu \gamma^5 ) & = b \\
  ( \Mw \Mw^\dagger + \cancel{i\mu\gamma^5 \Mw^\dagger - i \mu \Mw \gamma^5} + \mu^2 ) & = b \\
  ( \Mw \Mw^\dagger + \mu^2 ) & = b \, ,
  \end{split}
\end{equation}
where in the last line $\gamma_5$-hermiticity of $\Mw$ was used.
With the clover term, $T$, in the operator, the calculation goes through in the same way, with the result
\begin{equation}
  \begin{split}
    & ( \Mw \Mw^\dagger + \Mw T + T \Mw^\dagger + \cancel{i\mu\gamma^5 \Mw^\dagger - i \mu \Mw \gamma^5} + i\mu\gamma_5 T - i\mu T \gamma_5 + T^2 + \mu^2 ) = b \\
    & ( \Msw \Msw^\dagger + \mu^2 ) = b \, ,
  \end{split}
\end{equation}
where $\Msw = \Mw + T$.

The algorithm is listed below in Algorithm \ref{alg:cgm} (see also Ref.~\cite{Chiarappa:2006hz}
and references therein), with the identification $\sigma_k = \mu_k^2$.
Note that in line 6 below, $\alpha_{n-1}(1+\sigma_k\alpha_n)$ is correct, in contrast to Ref.~\cite{Chiarappa:2006hz}.

\begin{algorithm}
  \caption{CGMMS algorithm}
  \label{alg:cgm}
  \begin{algorithmic}[1]
    \vspace{.2cm}
    \STATE $n=0, x_0^k = 0, r_0 = p_0 = p_0^k = b, k_\mathrm{max},
    \delta, \epsilon$
    \STATE  $\biggl.\biggr.\alpha_{-1} = \zeta_{-1}^k = \zeta_0^k = 1, \beta_0^k = \beta_0 = 0$
    \REPEAT
    \STATE $\alpha_n = (r_n, r_n) / (p_n, A p_n)$
    \FOR{$k = 1$ to $k_\mathrm{max}$}
    \STATE $\biggl.\biggr.\zeta_{n+1}^k = (\zeta^k_n  \alpha_{n-1}) / 
      (\alpha_n \beta_n(1 - \zeta_n^k / \zeta^k_{n-1}) + \alpha_{n-1}
      (1+\sigma_k\alpha_n))$
    \STATE $\alpha^k_n = (\alpha_n \zeta_{n+1}^k)/ \zeta_n^k$
    \STATE $\biggl.\biggr.x_{n+1}^k = x_n^k + \alpha_n^k p_n^k$
    \IF{$\|\alpha^{k_\mathrm{max}} p^{k_\mathrm{max}}\| < \delta$}
    \STATE $k_\mathrm{max} = k_\mathrm{max} -1$
    \ENDIF
    \ENDFOR
    \STATE $x_{n+1} = x_n + \alpha_n p_n$
    \STATE $\biggl.\biggr.r_{n+1} = r_n - \alpha_n Ap_n$
    \STATE $\beta_{n+1} = (r_{n+1}, r_{n+1}) / (r_n, r_n)$
    \STATE $\beta_{n+1}^k = \frac{\beta_{n+1} \zeta_{n+1}^k \alpha_n^k}{\zeta_{n}^k\alpha_n}$
    \STATE $\biggl.\biggr.p_{n+1}^k = \zeta_{n+1}^k r_{n+1} + \beta_{n+1}^k p_n^k$
    \STATE $n=n+1$
    \UNTIL{$\|r_n\|<\epsilon$}
  \end{algorithmic}
\end{algorithm}

The implementations in \texttt{solver/cg\_mms\_tm.c} and \texttt{solver/cg\_mms\_tm\_nd.c} use a slightly different approach in that the lowest shift is included in the operator $A$, such that the higher shifts are $\sigma_k-\sigma_0\, \forall k > 0$.

It should be noted that $\sqrt{\sigma_k}$ are passed to \texttt{cg\_mms\_tm} and the solver internally squares these and shifts them by $\sigma_0$.

\subsubsection{Single flavour Wilson (clover) fermions in the rational approximation}

For details about the rational approximation in tmLQCD, see Section~\ref{subsec:rationalhmc}.

\textbf{QPhiX interface:} For the HMC with a single flavour of Wilson (clover) fermions (\texttt{RAT} or \texttt{CLOVERRAT} monomials), the function \texttt{solve\_mshift\_oneflavour} of \texttt{solver/monomial\_solve.c} provides a wrapper for tmLQCD or external multi-shift solvers.

Note that it passes the shifts as expected by tmLQCD's \texttt{cg\_mms\_tm}, which means that they need to be squared.
For the QPhiX normalisation, the QPhiX solver interface also divides them by $4\kappa^2$.
The shifts are taken as is and not shifted by $\sigma_0$.

\subsubsection{Two flavour Wilson twisted mass (clover) fermions in the rational approximation}

\textbf{QPhiX interface:} For the HMC with non-degenerate twisted mass (clover) doublets (\texttt{NDRAT} and \texttt{NDCLOVERRAT} monomials, exactly the same approach is used.

%%% Local Variables: 
%%% mode: latex
%%% TeX-master: "main"
%%% End: 
